%%%%%%%%%%%%%%%%%%%%%%%%%%%%%%%%%%%%%%%%%
% Short Sectioned Assignment
% LaTeX Template
% Version 1.0 (5/5/12)
%
% This template has been downloaded from:
% http://www.LaTeXTemplates.com
%
% Original author:
% Frits Wenneker (http://www.howtotex.com)
%
% License:
% CC BY-NC-SA 3.0 (http://creativecommons.org/licenses/by-nc-sa/3.0/)
%
%%%%%%%%%%%%%%%%%%%%%%%%%%%%%%%%%%%%%%%%%

%----------------------------------------------------------------------------------------
%	PACKAGES AND OTHER DOCUMENT CONFIGURATIONS
%----------------------------------------------------------------------------------------

\documentclass[paper=a4, fontsize=11pt]{scrartcl} % A4 paper and 11pt font size

\usepackage[T1]{fontenc} % Use 8-bit encoding that has 256 glyphs
\usepackage{fourier} % Use the Adobe Utopia font for the document - comment this line to return to the LaTeX default
\usepackage[english]{babel} % English language/hyphenation
\usepackage{amsmath,amsfonts,amsthm} % Math packages
\usepackage{graphicx} % figures
\usepackage{listings} % Required for insertion of code

\usepackage{lipsum} % Used for inserting dummy 'Lorem ipsum' text into the template

\usepackage{sectsty} % Allows customizing section commands
\allsectionsfont{\centering \normalfont\scshape} % Make all sections centered, the default font and small caps

\usepackage{fancyhdr} % Custom headers and footers
\pagestyle{fancyplain} % Makes all pages in the document conform to the custom headers and footers
\fancyhead{} % No page header - if you want one, create it in the same way as the footers below
\fancyfoot[L]{} % Empty left footer
\fancyfoot[C]{} % Empty center footer
\fancyfoot[R]{\thepage} % Page numbering for right footer
\renewcommand{\headrulewidth}{0pt} % Remove header underlines
\renewcommand{\footrulewidth}{0pt} % Remove footer underlines
\setlength{\headheight}{13.6pt} % Customize the height of the header

\numberwithin{equation}{section} % Number equations within sections (i.e. 1.1, 1.2, 2.1, 2.2 instead of 1, 2, 3, 4)
\numberwithin{figure}{section} % Number figures within sections (i.e. 1.1, 1.2, 2.1, 2.2 instead of 1, 2, 3, 4)
\numberwithin{table}{section} % Number tables within sections (i.e. 1.1, 1.2, 2.1, 2.2 instead of 1, 2, 3, 4)

\setlength\parindent{0pt} % Removes all indentation from paragraphs - comment this line for an assignment with lots of text

\lstset{
   extendedchars=true,
   basicstyle=\footnotesize\ttfamily,
   showstringspaces=false,
   showspaces=false,
   numbers=left,
   numberstyle=\footnotesize,
   numbersep=10pt,
   tabsize=4,
   breaklines=true,
   showtabs=false,
   captionpos=b
}

%----------------------------------------------------------------------------------------
%	TITLE SECTION
%----------------------------------------------------------------------------------------

\newcommand{\horrule}[1]{\rule{\linewidth}{#1}} % Create horizontal rule command with 1 argument of height

\title{	
\normalfont \normalsize 
\textsc{Udacity Data Analyst Nanodegree} \\ [25pt] % Your university, school and/or department name(s) % OVERRIDE
\horrule{0.5pt} \\[0.4cm] % Thin top horizontal rule
\LARGE{Project II} \\ % The assignment title % SIZE OVERRIDE
\horrule{0.5pt} \\[0.5cm] % Thick bottom horizontal rule
}

\author{Ming Wen} % Your name % OVERRIDE

\date{\normalsize\today} % Today's date or a custom date

\setcounter{section}{-1} % Section number start from 0

\begin{document}

\maketitle % Print the title

%----------------------------------------------------------------------------------------
%	PROBLEM 0
%----------------------------------------------------------------------------------------

\section{References}

\medskip
\noindent\fbox{
	\parbox{\textwidth}{
        Please include a list of references you have used in the project.
	}
}
\medskip

\begin{enumerate}
	\item https://en.wikipedia.org/wiki/Student\%27s\_t-test
	\item http://stackoverflow.com/questions/3505701/r-grouping-functions-sapply-vs-lapply-vs-apply-vs-tapply-vs-by-vs-aggrega
\end{enumerate}

%----------------------------------------------------------------------------------------
%	PROBLEM 1
%----------------------------------------------------------------------------------------

\section{Statistical Test}

\medskip
\noindent\fbox{
	\parbox{\textwidth}{
        Which statistical test did you use to analyze the NYC subway data? Did you
		use a one-tail to a two-tail P value? What is the null hypothesis? What is
		your p-critical value?
	}
}
\medskip

The null hypothesis $H_0$: The average of total number of riders in NYC subway
in raining days is the same as the average of that in non-rainy days.

Since we have no idea of population mean, depending on if it is raining,
an independent two-sample t-test is used to analyze the data.

From the null hypothesis, the two-tail P value is used. To simplify the problem
we assume that two samples have equal variance. Then the degree of freedom is given by
\begin{equation}
	df = n_1 + n_2 - 2 = 29
\end{equation}

Thus the p-critical value is $p_{critical} = \pm 2.045$ if the significant level is
$\alpha = 0.05$.

\medskip
\noindent\fbox{
	\parbox{\textwidth}{
        Why is the statistical test applicable to the dateset? In particular, consider
		the assumptions that the test is making about the distribution of ridership in two
		samples.
	}
}
\medskip

An independent two-sample t-test is applicable to the dataset for the following reasons:
\begin{itemize}
	\item We have a plenty of samples for the number of riders in both raining and
	non-rainy days. The mean of those samples follows normal distribution.
	\item We have no information about population means. Thus a t-test should be
	conducted rather than a z-test.
	\item Our hypothesis is defined as the equality of two sample means, so the
	two-tail test is applied in the task.
\end{itemize}

\medskip
\noindent\fbox{
	\parbox{\textwidth}{
        What results did you get from this statistical test? These should include
		the following numerical values: p-values, as well as the means for each of
		the two samples under test.
	}
}
\medskip

If define a raining day as the number of records whose $rain = 1$ is more than
the number of record whose $rain = 0$. Then there are 7 raining days and 24
non-rainy days in the dateset. Their corresponded statistics are as follows.

\begin{align}
	\bar{x}_{rain} &= 2830749 \nonumber \\
	\bar{x}_{norain} &= 2526914 \nonumber \\
	s_{rain} &= 632147 \nonumber \\
	s_{norain} &= 729616.8 \nonumber
\end{align}

The pooled standard deviation is
\begin{equation}
	s_{pool} = \sqrt{\frac{6*s_{rain}^2+23*s_{norain}^2}{29}} = 710548.5
\end{equation}

So the t value is
\begin{equation}
	t = \frac{\bar{x}_{rain} - \bar{x}_{norain}}{s_{pool} * \sqrt{\frac{1}{7}+\frac{1}{24}}}
	  = 0.995
\end{equation}

\medskip
\noindent\fbox{
	\parbox{\textwidth}{
        What is the significance and interpolation of these results?
	}
}
\medskip

Under significant level $\alpha = 0.05$, $p_{critical} = \pm 2.045$.
Since $t < \vert p_{critical} \vert$, we shouldn't reject the null hypothesis.
It means that the average of total number of riders in NYC subway
are the same in raining days and non-rainy days.

%----------------------------------------------------------------------------------------
%	PROBLEM 2
%----------------------------------------------------------------------------------------

\section{Linear Regression}

\medskip
\noindent\fbox{
	\parbox{\textwidth}{
        What approach did you use to compute the coefficients theta and prediction
		for $ENTRIESn_hourly$ in your regression model:
		
		(a) OLS using Statsmodels or Scikit Learn
		
		(b) Gradient descent using Scikit Learn
		
		(c) Something different
	}
}
\medskip

Answer of 2.1

\medskip
\noindent\fbox{
	\parbox{\textwidth}{
        What features (input variables) did you use in your model? Did you use any
		dummy variables as part of your features?
	}
}
\medskip

Answer of 2.2

\medskip
\noindent\fbox{
	\parbox{\textwidth}{
        Why did you select these features in your model? We are looking for specific
		reasons that lead you to believe that the selected features will contribute
		to the predictive power of your model.
		
		\begin{itemize}
			\item Your reasons might be based on intuition.
			\item Your reasons might also based on data exploration and experimentation.
		\end{itemize}
	}
}
\medskip

Answer of 2.3

\medskip
\noindent\fbox{
	\parbox{\textwidth}{
        What are the parameters (coefficients) of the non-dummy features in your
		linear regression model?
	}
}
\medskip

Answer of 2.4

\medskip
\noindent\fbox{
	\parbox{\textwidth}{
        What is your model's $R^2$ value?
	}
}
\medskip

Answer of 2.5

\medskip
\noindent\fbox{
	\parbox{\textwidth}{
        What does this $R^2$ value mean for the goodness of fit for your regression
		model? Do you think this linear model to predict ridership is appropriate
		for this dataset, given this $R^2$ value?
	}
}
\medskip

Answer of 2.6

%----------------------------------------------------------------------------------------
%	PROBLEM 3
%----------------------------------------------------------------------------------------

\section{Visualization}

\medskip
\noindent\fbox{
	\parbox{\textwidth}{
		Please include two visualizations that show the relationships between two or
		more variables in the NYC subway data. Remember to add appropriate titles
		and axes labels to your plot. Also please add a short description below each figure
		commenting on the key insights depicted in the figure.
		
		One visualization should contain two histograms: one of $ENTRIESn_hourly$ for
		rainy days and one of $ENTRIESn_hourly$ for non-rainy days.
		
		The other can be more freedom.
	}
}
\medskip

Answers here.

%----------------------------------------------------------------------------------------
%	PROBLEM 4
%----------------------------------------------------------------------------------------

\section{Conclusion}

\medskip
\noindent\fbox{
	\parbox{\textwidth}{
		From your analysis and interpretation of the data, do more people ride the NYC
		subway when it is raining or when it is not raining?
	}
}
\medskip

Answer of 4.1

\medskip
\noindent\fbox{
	\parbox{\textwidth}{
		What analyses lead you to this conclusion? You should use results from both
		your statistical test and your linear regression to support your analysis.
	}
}
\medskip

Answer of 4.2

%----------------------------------------------------------------------------------------
%	PROBLEM 5
%----------------------------------------------------------------------------------------

\section{Reflection}

\medskip
\noindent\fbox{
	\parbox{\textwidth}{
		Please discuss potential shortcomings of the method of your analysis, including:
		\begin{enumerate}
			\item dataset
			\item analysis, such as the linear regression model or statistical test.
		\end{enumerate}
	}
}
\medskip

Answer of 5.1

\medskip
\noindent\fbox{
	\parbox{\textwidth}{
		(Optional) Do you have any other insight about the dataset that you would like to share
		with us?
	}
}
\medskip

Answer of 5.2

\end{document}


